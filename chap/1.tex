\chapter{The Gods Have Fallen Onto the Earth}
\fancyhead[RE,LO]{Chapter 1}
\label{ch:1}
% 5 pages

Here we are, then, infected with an incurable Good. This millenium ends with a honeymoon. Humankind is on vacation. It is as a vast leisure park that I wanted to try and paint our planetary village. A park the size of a territory, of France, of Europe, of the globe perhaps. A big, spontaneous fair, with its own quarters, its long avenues, its particular attractions, its sketches, games, parades, screenings, its crises of love, of indignation\dots~

To understand the end of our century\footnote{\textbf{Translator's note:} this work was originally published in October 1998.}, we must first visit it, allow ourselves to be taken away by its currents, to not be afraid of the mobs, to clap along with the wolves, to be at unison with its euphorias. It is by wondering along booths that we can hope to understand it. Let us not hesitate any longer! Let us not be afraid! Let us dance together! All the games are offered to us! It is evasion! A Pasha's\footnote{\textbf{Translator's note:} essentially the title of a nobleman of the Ottoman empire.} life! Florida! Wonderland! California! The world is a factory of pleasures! And with fanfare! In full gaiety! Onwards with our fantasy!

\begin{displayquote}
	\textit{
		``How glorious it is to embark upon a new career, and to appear all of a sudden in the learned world, a book of discoveries in hand, like an unexpected comet sparkling through space!''
	}
\end{displayquote}

Thus exclaimed Xavier de Maistre in the first pages of his \textit{Voyage autour de ma chambre}.\footnote{\textbf{Translator's note:} \textit{``Travels Around My Bedroom''}.} An unexpected comet\dots~ But it is not the point, here, to propose discoveries. A simple promenade through what we live each day, or what we believe we live, what we love or what we dread, would teach us a thousand times more. Indeed, it is like a great theme park that we must visit the spirit of time. With its displays and its reflections, its fifteen-minutes-of-famers, its false roads of false cities of everywhere, its rebuilt castles, its excitants, its mounted pieces, its decorative pieces made of synthetic resin, its anonymous actors going about their business, and, under appropriate customes, simulating their customary tasks\dots~ there are no more engimas, no more mysteries. It's not worth it anymore to be tired. The Good is the anticipated answer to all the questions that we no longer ask. Blessings rain from everywhere. The Gods have fallen onto the Earth. All causes are heard, there exists no presentable alternatives to democracy, to the couple, to human rights, to the family, to tenderness, to communications, to charges on labor, to the homeland, to solidarity, to peace. The last visions of the world were taken off the walls. Doubt has become a sickness. The incredulous prefer shutting themselves up. Irony feels small. Negativity huddles up. Death itself doesn't live large, knowing that it is not for long under the scalding sun of the triumph of life expectancy.

Certainly, some old ruins encumber us, some vague souvenirs of past wars; we're going to have to sweep them out, it is only a question of days, of weeks. Already psychoanalysis, Marxism, have been consigned to oblivion, dumped in the trash, liquidated like Ozone layer-puncturing aerosols, the moment we noticed that these disciplines served neither to heal myopaths or even to save the ice caps.

And it is but the beginning of our grand spring cleaning. No more nostalgic longings! Long live the Festival! Forgetfulness in joy! This era shows itself full, but only an ingrate would dare complain. Particles that we are! Fragments! We owe everything to our multitude. All that exists exists only under the condition that it diffuses itself to the largest number; the maximum number of copies; at the most propicious hour. Everything, literally: me, you, objects. \textit{Prime time} has petrified time, replaced the hours and the seasons. Going it alone is out of question. Survival only is fair enough. Subsist, then tell.

The sunken scenes of History are no longer led across the boardwalks for us to reflect in, to wonder between ourselves how such barbarities were once possible. Onwards with the music! Look alive! Just like in \textit{Voyage}, near the end\dots~

\begin{displayquote}
	\textit{
		``Bim et Boum! Et Boum encore! Et que je te tourne! Et que je t’emporte! Et que je te chahute! Et nous voilà tous dans la m\^el\'ee, avec des lumi\`eres, du boucan, et de tout! Et en avant pour l’adresse et l’audace et la rigolade! Zim!''
	}
\end{displayquote}

Hold fast, get on this Western train, it is just about to depart. Or do you prefer the thrill of a rollercoaster ride? Three kilometers of ups and downs at more than a hundred kilometers an hour. Look alive, damn it! Discover your third wind! Great adventures await us!

We've been set free. It's done. No more worries. Nowhere. Pluralist democracy and the market economy are taking care of us. The rest is ancient history. Listen to your body! Tone your muscles! All the pleasures of Polynesia are at your fingertips. Discover the aquatic gymnasium. Become a backpacker under the bamboos. Attack the Inca temple made from papier-mach\'e. Climb the bubble volcano. Find the villains hiding within it. Chase out our \textit{real} enemires, those last hideous tyrants, there, clearly visible, center page, precious vestiges of lost causes, atrocious, ultimate villains.

Ah! The system does it right! There will be something for everyone. The Good, the wholesome Good, against all Evil! To the end! Herein lies the epic. Everything that is definitely right against everything that is wrong forever. The New Goodness has the wind in its sails against sexism, against racism, against discrimination in all its forms, against animal abuse, against ivory and fur trafficking, against those responsible for acid rain, xenophobia, pollution, the massacre of landscapes, smoking, the Antarctic, the dangers of cholesterol, HIV/AIDS, cancer and so on.

Mourning evil is a hell of a job, that much is certain. Especially since the devil dons masks and hides under rubbish bins. Where did he go again, that one? In which black hole blacker than him? One could believe oneself in a great strange struggle, without real adversaries; in a great affirmation to be repeated, rehashed, consolidated over and over again, and all the more so as it has no obvious opposite\dots~ But all the more reason! Let's get on with it!  We need strong emotions. Where else would we find them if not through our memories in imitation, in retrospectives, in reminders? Ghosts of culprits to chase out! Put in some elbow grease! You're not too scared, are you? See you at the gate then; let's climb into that red wagon. Feet stalled, hands clutched, we're at the start of the infernal convoy, in all its colors. The voluptuousness of horror in its purest form, with a stomach of mush, a heart at a hundred and fourty beats per minute, a leap into the void, all up there, on loop-the-loops of three hundred and sixty degrees in the midst of cries of panic\dots~

Having said that, don't make me imply what I will never write. The magic formula today, if one ever wishes to hope for peace, consists in declaring from the outset that one has nothing against anyone, most of all against those one attacks. This is indispensable. \textit{``The author would like to point out that characters, places, events, have no relation to reality\dots''}

So it goes without saying that I'm definitely for all good causes, and definitely against bad ones. That's it. There you have it. And it gets a lot better when you say it. I'm for all that can be good and against all that is bad. I'm for transparency and against opacity. For truth against error. For authenticity against lies. For reality against decoys. For morality against immorality. So that everyone may eat their fill, so that there may be no more outcasts anywhere, so that dieticians may triumph.

Don't make me pretend things.

In the midst of this flood of bounties that fills us from all sides, it is only the destiny of Evil which seems to me instructive. It is its becoming, its future\dots~ Where could he have slipped into? Which hatch? Who supports the rumors?  Who is blowing the scandal out of the air? Where do the pleasures of hell crawl?  Who barks real horrors still?  All I see everywhere is politeness, discreet approaches, flattery, pettiness and camouflage\dots~ Great sprinkles of holy water\dots~ In order not to fall into the trap, philosophers in Italy have even tried to invent a new ideology without danger, a new conceptual Schmilblick made of bits of Nietzsche or Heidegger watered down to the last detail: ``weak thinking'', they call it. Weakism. It's touching. Finally a vision of the world without dyes! Not one idea that's bigger than the other! In France itself, the current President\footnote{Fran\c{c}ois Mitterand.} had to have his teeth filed in order to get to where we can see him; no one wanted him as long as he was wearing his vampiric canines.

All antagonisms devoid of their substance have been given new clothes for the parade. Certificates of good living and good will are work like socks, impossible to fully hide. Even the racists today want to be anti-racist like everyone else; they keep sending their own disgusting obsessions back to others. ``It's you! -- No, it's you! -- Not at all!''  We don't know who's playing which role anymore. The audience is there, waiting, hoping for a beating, screaming, wanting something to happen. Boredom lurks, invades everything, depressions multiply, the quality of the scriptwriting goes down, suicide rate skyrockets, bad hygiene drips everywhere, it's the invasion of puerile grace, it's the great \textit{Gala du Show du Cœur}.

Bernard de Mandeville, who got himself into a lot of trouble for trying to show that it is often the worst scoundrels who contribute to the common good, already noted in the 18th century in his \textit{Fable des abeilles}:

\begin{displayquote}
	\textit{
		``One of the main reasons why so few people understand themselves is that most writers spend their time explaining to people what they should be, and almost never go to the trouble of telling them what they are.''
	}
\end{displayquote}

We understand these people. If they did the opposite, poor things, they would never be let out of prison.