\chapter*{Preface: The Childhood of the Good}
\fancyhead[RE,LO]{Preface}
\label{ch:preface}
% 8 pages

The Good moves swiftly. It gallops forward. It climbs from all sides. It festers, grows, conquers terrain, recruits new missionaries at every opportunity. The Good grows rapidly -- little by little, permeating all exits and forbidding all escape routes. It is the Good that remakes the day and the night, the sun and the stars, space and time. Since the dawn of \textit{The Empire of the Good}, the Good has spiraled downwards. Seven little years were enough for the Good to sink, to stampede, to unfurl itself, carrying and dragging along with it all that it could find in its path, overturning any resistance that remained, overflowing from its bed, grazing its banks, hurtling on a train from hell -- or rather, from paradise -- spreading everywhere, blossoming, circulating, conquering and subjugating all that could still be tempted to oppose it.

Now, the Good has reached its objective, or almost -- and it loses itself deliciously in the immensity of its Festival, like a river into the sea that was long promised to it. And all that it has uprooted in its mad race, it now offers to the endless backwash in which it itself falls into ruin, like so many witnesses of their own common victory.

Together, now, the Good and the Festival, their powers combined, know no limit; and they melt into each other, to start, based on the invented power of their pretended enemies, the virulent dishonesty and archaic malfeasance of which these good apostles never cease to denounce. The Good, like the Festival, are ticklish, susceptible, irritable. They feed themselves on the sentiment of persecution. Having reduced all opposition to mutism does not suffice; they must still ceaselessly shake the scarecrow. And in the widespread silence of cowardice, of stupefaction and acquiescence, they must always fortify themselves against phantom attacks, puppet perils and adversarial simulacra.

In 1991, the Good was, for all intents and purposes, still in its childhood. It was still far from knowing the full extent of its powers. Nevertheless, it tried out its strengths, with the air of a hestitating infant; stammering, like a toddler that, certainly, was already monstrous and that benefitted from a concerningly strong bill of health, but with whom we could still envision a quelling through an accident, an illness, a cot death, something or another that could save humanity from the fatal peril of its rapid growth, from the weight of its irresistible extension.

In 1991, the Good seemed fragile, like a simple hypothesis, akin a supposition to which it was sufficient to cast a puzzled look at the right moment to discourage the worst of our kind from pursuing. The Good seemed timid, emotive, anxious vis-\'a-vis the sniggers that its initial philantrophic abuses could set off within a few free spirits, which indeed roamed free but were already under a state of precarious survival. And Cordicopolis, the city of rose-tinted nightmares which the Good was, to the applause of almost all, laying the foundations, still had only the allure of a half-baked utopia.

The Good, in 1991, was still in swaddling clothes, but this little Nero, this dictator of Altruism, already had serious assets on its side. It had already begun to expand its radiant prison around humanity and with humanity's consent. All of its predecessors, under the name of, for example, \textit{public welfare} -- with whatever this notion impliles in terms of an undifferentiated whole whose growth should be encouraged through the police, the justice system, and of course the prattle of the media, all only waiting to flourish thanks to it and to impose themselves in all areas of everyday existence. All that was left for the Good was for it to stream into the grand river of \textit{Amour-en-liesse}; and to admit into it the idea that the virtuous life is the festive life. The Good had flown straight in that direction. It had quickened itself, hasty, precipitously. It had only one goal, and it was clearly in approach.

In a general sense, most of the themes that I discussed in 1991 have not ceased to worsen, to darken still, even as they appear in colors that are more and more desirable to populations at large. The cordicoles\footnote{\textbf{Translator's note:} from the latin \textit{cor}, \textit{cordis} (``heart'') and \textit{-cole} (``he who veneres''), meant in a lightly deprecating tone.}, cordicolicians and cordophiles multipiled. Cordicologists, on the other hand, were not legion. And cordicoclasts (by which I mean the eventual demystifiers of cordicole norms) observed in silence. They observe in silence still. Since 1991, the actors of Transparency, those possessed by Homogeinity, the cursaders for the abolition of all differences and the raging bulls of retroactive trials launched themselves with a frenzy the good merits of which no one still dares to doubt. Operation ``Clean Record'' is almost accomplished. The demand for laws, which I at the time was still only sketching the pathology, and that I was then brought to define as a sort of \textit{penal envy}, had still not found its good rhythm, its groove, it still hadn't become the cry of extasy and resentment of millions of human ants to which judges, galvanized by the encouragement of the media herd offering the spectacle of the daily grind of their politicians. The demand for laws was also still not precisely the powerful accelerator of the \textit{change of attitude}\footnote{\textbf{Translator's note:} \textit{changement des mœurs}, which can also be translated as ``change of lifestyle'', ``change of behavior'', or ``change of habit'', especially moral habit.} that it was henceforth destined to be; neither was it still the ideal machine for criminalizing, with all its strength, all which did not have the ability or possibility to present itself \textit{on time} as a \textit{long-standing victim}. We had not yet, in 1991, opened the floodgates to the radiant wrongdoers of the penal code, nor to the squared knives of judiciary power. We had not quite yet, in 1991, changed the meaning of words to the point of seeing, without blinking, the worst consensual scoundrels fighting consensus, and the strongmen of neo-conformism rise with indignation against conformism. In 1991, it was still possible to be astonished at the spectacle rendered by so many beautiful souls going to battle (for the good causes), with an ardour that one could have seen, in other times, employed within more paradoxical, more perilous, more equivocal, and therefore more interesting theatres.

In 1991, those that I would call, a few years later, the \textit{truismocrats} (those men and women that fill with all the pathos in the world their fight against asbestos, paedophilia, smoking, homophobia, xenophobia, having replaced the great fights-to-the-death of yesteryear with a dutiful humanitarian imposition to which they give the appearance of a perpetual crusade) had not yet set up their regular patrols, watching so that none remains estranged from their indefatigable exploits. In 1991, all of the \textit{bondieuseries}\footnote{\textbf{Translator's note:} a church ornament or devotional object, especially one of little artistic merit (i.e. a ``religious knick-knack''). Commonly found at gift shops inside large and well-known churches in Europe.} of generalized ``creolization'', that shepherd's idyll in the form of a new age archipelago, had not yet accomplished the ensemble of their work towards unification, but nevertheless employed themselves assiduously towards it. The Positive, in 1991, did not yet parade itself without interruption, without ever again confronting itself with the Negative, the ``resurgence'' of which it still did not cease to denounce, because it is exactly that denounciation that kept the Positive alive, at the same time as it allowed it to pursue its long fight of the obvious, its epic of pleonasm.

Nothing of what I have just described was quelled. All of it still seemed to be very much in play. The Empire, since then, had envenomed itself. It's what it knew how to do with the greatest talent. And the sexual adventure, for example, of which I sketched the requiem because I foresaw it henceforth employable only in the past tense, seems to be a settled affair: it has succumbed indefinitely to the undifferentiating propaganda of the institutional sexual movement \textit{de masse}, which entertains almost as much of a relationship with the sexuality of the individual as an ice cube with a river trout. On that point, and at the end of a few millenia of human history in which one was necessarily, by definition, guilty, it sufficed, to close the entire question in five minutes, to convince oneself that too great of an interest towards sexual \textit{difference} was the source of all crimes, and that hierarchical differentiation, it itself the generator of ``inequalities and exclusions'', flowed directly as a result of it.

The Good strode quickly. The Good struggled onward. It toiled well. In its mad rush, it even succeeded in reining in the Evil. It took it way. It converted it. It hogged it. It had it in its pocket. It literally expropriated it, captured it. All of this in order to ultimately present it as dowry at the moment of triumphal convention with the Festival. Because the Good, all things considered, had united with the Festival; and it is from that conjoint fusion of these two ``values'' that the most extraordinary new facts of the last few years blossomed forth. The Good had gotten married. It couldn't be said better. And if, today, my \textit{Empire} seems to sometimes evoke events that could have occurred a century ago, it is that between then and now, the resulting baby has grown, has forced and enforced, driven by all sides it spread out, it went overboard. It became an adult. It emancipated itself. It unchained itself. Sole heir of the Evil (through the suppression or deletion of the latter), it can at once declare it an outlaw while still picking out the useful crumbs for itself. The Negative, which it abhorred precisely because it represented the power of historical development, was sequestered. And, so that the fates of preceding societies never befell it, i.e. to one day be revealed as a rotting state of affairs, the child imagined (less stupidly, less na\"ively than its predecessors in oppression) to integrate itself as an ornamental antidote to the Negative. In order to never risk revealing this resulting double negative (in the way that the bourgoisie, for example, risks revealing the proletariat), it resolved to raise it hidden away, fed with the grafter's baby bottle. The Good apes the Evil whenever necessary. It stokes centers of conflict like campfires. And the new generations of synthetic rebels that it fabricated have no risk of revealing themselves one day as grave-diggers, as successors, and even less as usurpers or demolishers of their exemplary employer.

The Good has worked its ass off. It did a good job. In advance, it sterilises all objections, all subversions, all contestations that could be raised. Or rather, it enlists them. It recruits them, and puts to them to the service of the perpetual Festival; of which it would be impious by now, even dangerous to deny its virtues -- its domesticating, grinding, crushing, civilizing virtues.

The Good has raced, has precipitated. It has reached its goal, attained its desire. And it is in the process of obtaining that of which no institution, no power, no terrorism of the past, no police, no army has ever been able to obtain: the spontaneous adhesion of almost the entire public interest; that is to say, the enthusiastic amnesia from each individual against their own private interest, or even its sacrifice. Nothing in past history, except perhaps for the furious mobilization of the Germans and the French, their heads rearing at the declaration of war in 1914, and, correlatively, the abrupt muteness of those who (anarchists, pacifists, social democrats) should have been opposed to that general dementia, can ever approach such a tremendous public approval.

Within the Good-turned-into-Festival, the remains nothing but the Good, there remains nothing but the Festival; and all the other contents of our existence have almost completely melted through contact with the sheer heat of that fire. The Empire says now, paraphrasing Hegel: ``all that is real is festive, all that is festive is real.''

In a society where transgression and rebellion became routine, where non-conformism has its payroll and where anarchism is gilded on its edges, it became logical to recognize within its festive masses, linked for all eternity to the transgression and to the ritual violation of the norms of everyday life, the justificatory apotheosis of its existence. Except that there are no more norms, no more everyday life; and by extending the whole of existence to the Festival, which was hitherto ephemeral disorder and the reversal of taboos but which had now become not just the norm, but had even assumed the role of the police. And this problem wouldn't have existed, were it not for the pen-pushers and the correctional officers of this new hyperfestive society, and especially were it not for all means of comparison with the past having been extinguished at the same time.

The tomorrows that sing of ancient rebellions had become but empty promises, always broken by our todays that instead holler and thunder. Since there has been no more work, or rather, since workers are no longer as truly necessary to the smooth running of the planet as they once were, the eminent dignity that used to derive from work has been replaced by the eminent derision of the festive man. Stripped of any meaning, of any other goal than to assert its stupid pride, here is the pack such as in itself. What does it want? Nothing other than to become more numerous, therefore always prouder, more self-satisfied, more content with itself and with the universe. Our world is the first to have invented instruments of persecution or destruction powerful enough so that it is no longer even necessary to physically go and smash the windows or doors of the houses in which those who seek to exclude themselves from it, and are therefore its enemies, are hiding.

It is undoubtedly the greatest originality of this work that it does not suggest any solution to all of this which, under the aspect of an ever accelerating disaster, has ended up substituting society. I am sure that it will be a pleasure to note that, in 1991, I already saw no way out of this situation. It will also be observed, always with pleasure, that I was not bothered to convince those who had not already been exceedingly convinced by themselves of the pertinence of such a vision. One will be pleased to see that I do not envisage the slightest glimmer of hope in this electronic night where all the charlatans are grey and the merchants of illusions see life through rose-colored glasses on the Web.

It is a great misfortune to live in such abominable times. But it is an even worse misfortune not to try, at least once, for the beauty of the gesture, to grab them by the throat. Before moving from speech to action, or from thought to concrete examination, from essay to novel, that is to say, to the auscultation of what could subsist of autonomous existence in the conditions of survival of this planetary city that I had previously baptized Cordicopolis but that must henceforth be called Carnivalgrad: herein ends the Empire.

\begin{flushright}
	\textit{August 1998.}
\end{flushright}